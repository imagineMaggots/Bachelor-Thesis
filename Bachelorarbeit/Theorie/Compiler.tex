% Options for packages loaded elsewhere
\PassOptionsToPackage{unicode}{hyperref}
\PassOptionsToPackage{hyphens}{url}
\documentclass[
]{article}
\usepackage{xcolor}
\usepackage{amsmath,amssymb}
\setcounter{secnumdepth}{-\maxdimen} % remove section numbering
\usepackage{iftex}
\ifPDFTeX
  \usepackage[T1]{fontenc}
  \usepackage[utf8]{inputenc}
  \usepackage{textcomp} % provide euro and other symbols
\else % if luatex or xetex
  \usepackage{unicode-math} % this also loads fontspec
  \defaultfontfeatures{Scale=MatchLowercase}
  \defaultfontfeatures[\rmfamily]{Ligatures=TeX,Scale=1}
\fi
\usepackage{lmodern}
\ifPDFTeX\else
  % xetex/luatex font selection
\fi
% Use upquote if available, for straight quotes in verbatim environments
\IfFileExists{upquote.sty}{\usepackage{upquote}}{}
\IfFileExists{microtype.sty}{% use microtype if available
  \usepackage[]{microtype}
  \UseMicrotypeSet[protrusion]{basicmath} % disable protrusion for tt fonts
}{}
\makeatletter
\@ifundefined{KOMAClassName}{% if non-KOMA class
  \IfFileExists{parskip.sty}{%
    \usepackage{parskip}
  }{% else
    \setlength{\parindent}{0pt}
    \setlength{\parskip}{6pt plus 2pt minus 1pt}}
}{% if KOMA class
  \KOMAoptions{parskip=half}}
\makeatother
\setlength{\emergencystretch}{3em} % prevent overfull lines
\providecommand{\tightlist}{%
  \setlength{\itemsep}{0pt}\setlength{\parskip}{0pt}}
\usepackage{bookmark}
\IfFileExists{xurl.sty}{\usepackage{xurl}}{} % add URL line breaks if available
\urlstyle{same}
\hypersetup{
  hidelinks,
  pdfcreator={LaTeX via pandoc}}

\author{}
\date{}

\begin{document}

\subsubsection{Quellen}\label{quellen}

Drachenbuch Compilerbau (Aho)

\subsection{Wozu dient ein Compiler}\label{wozu-dient-ein-compiler}

Wenn wir über einen Compiler sprechen, meinen wir hiermit üblicherweise
ein Programm, welches eine Quellcode in einer für einen
Softwareentwickler verständlichen Hochsprache (e.g., C, C++, Perl, etc.)
in eine für einen Rechner verständliche Bitsequenz übersetzt. Allerdings
stellt dies nur eine Art eines Compilers dar. Es handelt sich nämlich
vielmehr um einen Überbegriff, denn eine solche eine solche Übersetzung
muss nicht zwangsweise zwischen Programmier- und Maschinensprache
erfolgen, sondern kann z.B. auch zwischen zwei Hochsprachen (beispiele
einfügen) erfolgen.

\subsection{Bau/Funktionsweise eines
Compilers}\label{baufunktionsweise-eines-compilers}

Die Übersetzung erfolgt auf Grundlage einer mehrstufigen Analyse.

\subsubsection{Lexikalische Analyse}\label{lexikalische-analyse}

\subsubsection{Syntaktische Analyse}\label{syntaktische-analyse}

\subsubsection{Semantische Analyse}\label{semantische-analyse}

\end{document}
