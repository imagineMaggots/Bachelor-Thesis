% exclude the two following
\documentclass{standalone}
\usepackage{epigraph}
\begin{document}
% exclude the two preceding


\chapter{Einleitung}
\epigraph{\enquote{\textit{Jedermann wird zugestehen, daß der Mensch ein soziales Wesen ist. Wir sehen es in seiner Abneigung gegen Einsamkeit sowie seinem Wunsch nach Gesellschaft über den Rahmen seiner Familie hinaus.}}}{\textit{Charles Darwin (1809-1892)}}

Wie Menschen sind nicht auf uns alleine gestellt. Auch wenn ein Jeder von uns diesen Fakt zwar zu einem anderen Zeitpunkt in seinem/ihrem/... Leben lernt, wird sich (hoffentlich) nie etwas an diesem ändern. Hierbei ist es unabdingbar kommunikationsfähig zu sein, insbesondere für Informatiker \cite{Soziale_Kompetenzen_fuer_Informatiker_Schumacher_2013}. 
Eine wesentliche, allgemein bekannte Hürde ist hierbei die Multilingualität auf unserem Planeten, also der Fakt, dass wir nicht alle ein und dieselbe Sprache sprechen. Gerade durch die informationstechnologische Revolution der vergangenen Jahre, bzw. Jahrzehnte können wir hierbei keine Umwege über z.B. Gestiken oder Mimiken gehen, denn diese gehen in textlicher Form verloren. 
Das bedeutet, dass eine Übersetzung zwischen mehreren Sprache unabdingbar ist. Verschiedene Sprachen existieren nicht nur für uns Menschen. Man kann eine Programmiersprache im Wesentlichen wie eine Schnittstelle betrachten, an welcher menschliche und binäre Sprache aufeinandertreffen. 
Sogenannte Hochsprachen (wie zum Beispiel C, C++, Java, Python, Perl oder Ruby, und etwaige Andere) sind für den Menschen lesbar und werden durch einen Compiler in - für einen Rechner interpretierbare - Binärsprache, also Folgen von Einsen und Nullen, übertragen. 

Aus Vorangegangenem wird allerdings schnell klar, dass auch in der Softwareentwicklung die Internationalisierung auf unserem Planeten Vor- und Nachteile birgt. Anders als beim Kochen, wo es bekanntlicherweise heißt, dass zu viele Köche die Suppe versalzen, gilt hier \enquote{je mehr, desto besser}. Das Vier-Augen-Prinzip sollte allgemein bekannt sein und damit verbunden der Fakt, dass es besser ist, noch eine zweite Person über seine Arbeit herüberschauen zu lassen, da dieser unter Umständen Fehler auffallen können, welche man selbst aufgrund von \enquote{Betriebsblindheit} übersehen hat. 
Dieses Prinzip lässt sich natürlich noch ausweiten. Mit über 20 Millionen, aus verschiedensten Teilen der Erde stammenden Softwareentwicklern (https://www.jetbrains.com/lp/devecosystem-data-playground/) muss hierbei zwischen Mensch und Maschine, Hochsprache und Maschinensprache, aber auch noch zusätzlich zwischen Mensch und Mensch übersetzt werden. 
Gerade da Hoch- und insbesondere Skriptsprachen so gestaltet sind, dass sie für den Menschen möglichst einfach lesbar und verständlich sind, werden Befehle hierbei oftmals durch englische Begriffe (z.B. If-Then-Else, For, While, ...) ausgedrückt. Für reine Rechnungen mit Zahlen mag dies kein Problem darstellen, möchte man aber mit Texten arbeiten, wie beispielsweise in der Formatierung von Dokumenten mit LaTeX, stößt man hier schnell auf ein Problem, da eine Übersetzung aus dem Englischen in eine andere Sprache die Exklusion der für die Funktion des Codes unabdingbaren Befehle bedingt.

Es existieren bereits erste Ansätze, welche auf Basis von künstlichen Intelligenzen (KI) arbeiten (Beispiele aus der Mail). Im Rahmen dieser Arbeit soll nun ein System entstehen, welches diese Funktionalität der Übersetzung vom Englischen ins Deutsch (und umgekehrt) für verschiedene LaTeX-Dialekte implementiert. 

% exclude the following
\end{document}
% exclude the preceding